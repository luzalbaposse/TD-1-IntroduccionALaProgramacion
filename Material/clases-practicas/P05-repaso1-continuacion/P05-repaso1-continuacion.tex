\newcommand{\docente}{Augusto Gonz\'alez Omahen}
\input{../init}
\usepackage{enumitem}
\begin{document}
\caratulaClase{Clase de repaso \grs{\it (continuaci\'on)}}

%%%%%%%%%%%%%%%%%%%%%%%%%%%%%%%%%%%%%%%%%%%%%%%%%%%%%%%%%%%%%%%%%%%%%%%%%%%%%%%%
\begin{frame}
\titlepage

\end{frame}

%%%%%%%%%%%%%%%%%%%%%%%%%%%%%%%%%%%%%%%%%%%%%%%%%%%%%%%%%%%%%%%%%%%%%%%%%%%%%%%%
\begin{frame}[fragile]\frametitle{Ejercicio 1}

Recordando un poco el algoritmo y su invariante de la clase pasada, vamos a demostrar que esa implementaci\'on es correcta respecto de su especificaci\'on.%\medskip


%\az{(a)} Funci\'on \vrd{\tt sumar\_digitos}:

\begin{columns} 

	\begin{column}{.68\textwidth}
\begin{lstlisting}[style=python]
def sumar_digitos(s:str) -> int:
  '''
  Requiere: s solo tiene digitos 
  Devuelve: la suma de los digitos en s
  '''
  res:int = 0
  i:int = 0
  #A
  while i < len(s):
    #B
    res = res + int(s[i])
    i = i + 1
    #C
  #D
  return res
\end{lstlisting}
	\end{column}

	\begin{column}{.28\textwidth}
		
		\begin{center}
			\medskip
			\vrd{\tt Invariante}\\
			\bigskip
			0 $\leq$ \textbf{i} $\leq$ len(s) \\
			\medskip
			\textbf{res} tiene la suma de los d\'igitos de s hasta la pos \textbf{i} (sin incluir)
		\end{center}		
	\end{column}%
\end{columns}

\end{frame}


%%%%%%%%%%%%%%%%%%%%%%%%%%%%%%%%%%%%%%%%%%%%%%%%%%%%%%%%%%%%%%%%%%%%%%%%%%%%%%%%

\begin{frame}\frametitle{Ejercicio 2}\small
	\vspace*{-1em}
	En cualquier palabra en espa\~nol, si a continuaci\'on de cada aparici\'on de una vocal se agrega primero una p y luego la misma vocal en min\'uscula, se obtiene la palabra en el idioma {\bf jeringoso}.\medskip
	
	Por ejemplo, {\it Elefante} es \az{\it Epelepenfapantepe}, {\it Tomate} es \az{\it Topomapatepe} y {\it Digital} es \az{\it Dipigipitapal}.\pause\bigskip
	
	Sea la siguiente especificaci\'on:\medskip
	
	\fcolorbox{black}{gray97}{\begin{minipage}{.95\textwidth}\footnotesize
			Traduce un texto sin tildes de espa\~nol a jeringoso.\\
			\az{Encabezado}: {\tt traducir\_a\_jeringoso(texto:str) $\rightarrow$ str}\\
			\az{Requiere}: {\tt texto} es una oraci\'on en espa\~nol sin tildes.\\
			\az{Devuelve}: la traducci\'on de {\tt texto} a jeringoso.
	\end{minipage}}\bigskip
	
	
	\begin{enumerate}[label=(\alph*)]
		\item Dada la implementaci\'on de la siguiente pagina, proponer un invariante.
		
		\item Dado el invariante, demostrar que la implementaci\'on es correcta respecto de su especificaci\'on.
		
		\item (para hacer en casa) Descargar el archivo \rj{\tt traductor\_jeringoso\_testing.py} y correr los tests sobre la implementaci\'on propuesta. En caso de encontrarse errores, corregir los test hasta que pasen todos.
	\end{enumerate}
	
\end{frame}


\begin{frame}[fragile]\frametitle{Ejercicio 2}

\begin{columns} 
	
	\begin{column}{.68\textwidth}
		\begin{lstlisting}[style=python]
def traducir_a_jeringoso(texto:str) -> int:
  '''
  Requiere: texto es en espaniol y sin tildes
  Devuelve: la traduccion de texto a jeringoso
  '''
  res:str = ''
  i:int = 0
  #A
  while i < len(texto):
    #B
    letra_actual:str = texto[i]
    res = res + letra_actual
    if es_vocal(letra_actual):
      res =res+'p'+letra_actual.lower()
    i = i + 1
    #C
  #D
  return res
		\end{lstlisting}
	\end{column}
	\pause 
	\begin{column}{.28\textwidth}
		
		\begin{center}
			\medskip
			\vrd{\tt Invariante}\\
			\bigskip
			0 $\leq$ \textbf{i} $\leq$ len(texto) \\
			\medskip
			\textbf{res} tiene la traducci\'on a jeringoso de texto hasta la pos \textbf{i} (sin incluir)
		\end{center}		
	\end{column}%
\end{columns}

\end{frame}

%%%%%%%%%%%%%%%%%%%%%%%%%%%%%%%%%%%%%%%%%%%%%%%%%%%%%%%%%%%%%%%%%%%%%%%%%%%%%%%%

\begin{frame}\frametitle{Recapitulando}

Cosas importantes:\bigskip

\begin{enumerate}

\item Tener en cuenta que con cada funci\'on que tenga ciclo hay que demostrar dos cosas: finalizaci\'on y correctitud.


\item Hacer una demostraci\'on no es contar l\'inea a l\'inea el algoritmo y tampoco son necesarios los ejemplos!


\end{enumerate}

\end{frame}


\end{document}
