\newcommand{\docente}{Cristian Nahuel Díaz}
\newcommand{\code}[1]{\texttt{#1}}

\input{init}


\usepackage{color}

\definecolor{mygreen}{rgb}{0,0.6,0}
\definecolor{mygray}{rgb}{0.5,0.5,0.5}
\definecolor{mymauve}{rgb}{0.58,0,0.82}

\lstset{
  backgroundcolor=\color{white},
  numbersep=10pt,
}


\begin{document}



\caratulaClase{Clase pr\'actica 7: M\'as listas, ciclos, invariantes.}

%%%%%%%%%%%%%%%%%%%%%%%%%%%%%

\begin{frame}
\titlepage
\end{frame}


%%%%%%%%%%%%%%%%%%%%%%%%%%%%%

\begin{frame}[fragile]\frametitle{Ejercicio 1 - Terminaci\'on}
\begin{columns}
\begin{column}{0.35\textwidth}
\lstinputlisting[language=Python]{suma-en-posiciones-impares}
\end{column}
\begin{column}{0.65\textwidth}
\begin{itemize}[<+->]
  \item \textcolor<1>{red}{i comienza valiendo 0}
  \item \textcolor<2>{red}{La lista $l$ no se modifica o sea que $len(l)$ es fijo}
  \item \textcolor<3>{red}{En cada iteraci\'on i se incrementa en 1}
  \item \textcolor<4>{red}{Por lo tanto es inevitable que en alg\'un momento $i = len(l)$, momento en el que la guarda evaluar\'a a False y el ciclo terminar\'a}
\end{itemize}
\end{column}
\end{columns}
\end{frame}

%%%%%%%%%%%%%%%%%%%%%%%%%%%%%

\begin{frame}[fragile]\frametitle{Ejercicio 1 - Correctitud}
\begin{columns}
\begin{column}{0.35\textwidth}
\lstinputlisting[language=Python]{suma-en-posiciones-impares}
\end{column}
\begin{column}{0.65\textwidth}

%% INVARIANTE
\fbox{%
  \parbox{\textwidth}{
    \I:
    \begin{itemize}[<+->]
      \pause
      \item $0 <= i <= len(l)$
      \item $len(vr) == i$
      \item Para j tal que $0 <= j <= i-1$, $vr[j] == l[j]$ si j es par y $vr[j] == l[j]+n$ si j es impar
    \end{itemize}
  }%
}

\end{column}
\end{columns}
\end{frame}


%%%%%%%%%%%%%%%%%%%%%%%%%%%%%

\begin{frame}[fragile]\frametitle{Ejercicio 1 - Correctitud}
\begin{columns}
\begin{column}{0.35\textwidth}
\lstinputlisting[language=Python]{suma-en-posiciones-impares}
\end{column}
\begin{column}{0.65\textwidth}

%% INVARIANTE
\fbox{%
  \parbox{\textwidth}{
    \I:
    \begin{itemize}
      \item $0 <= i <= len(l)$
      \item $len(vr) == i$
      \item Para j tal que $0 <= j <= i-1$, $vr[j] == l[j]$ si j es par y $vr[j] == l[j]+n$ si j es impar
    \end{itemize}
  }%
}
\\
\textcolor{blue}{Vale \I en (A)?}
\begin{itemize}[<+->]
  \item $i = 0$ entonces vale $0 <= i <= len(l)$
  \item $vr = []$ entonces vale $len(vr) == 0 == i$
  \item Como i = 0, el rango 0 a i-1 es vac\'io, entonces la 3er sentencia del invariante vale autom\'aticamente
\end{itemize}

\end{column}
\end{columns}
\end{frame}

%%%%%%%%%%%%%%%%%%%%%%%%%%%%%

\begin{frame}[fragile]\frametitle{Ejercicio 1 - Correctitud}
\begin{columns}
\begin{column}{0.35\textwidth}
\lstinputlisting[language=Python]{suma-en-posiciones-impares}
\end{column}
\begin{column}{0.65\textwidth}

%% INVARIANTE
\fbox{%
  \parbox{\textwidth}{
    \I:
    \begin{itemize}
      \item $0 <= i <= len(l)$
      \item $len(vr) == i$
      \item Para j tal que $0 <= j <= i-1$, $vr[j] == l[j]$ si j es par y $vr[j] == l[j]+n$ si j es impar
    \end{itemize}
  }%
}
\\
\textcolor{blue}{Asumamos que \I vale en (B). Sea K el valor de i}
\pause
\begin{itemize}
  \item Tenemos que $0 <= K \textcolor<3>{red}{<} len(l)$
  \item Vale $len(vr) == K$
  \item Para j tal que $0 <= j <= K-1$, $vr[j] == l[j]$ si j es par y $vr[j] == l[j]+n$ si j es impar
\end{itemize}

\end{column}
\end{columns}
\end{frame}

%%%%%%%%%%%%%%%%%%%%%%%%%%%%%

\begin{frame}[fragile]\frametitle{Ejercicio 1 - Correctitud}
\begin{columns}
\begin{column}{0.35\textwidth}
\lstinputlisting[language=Python]{suma-en-posiciones-impares}
\end{column}
\begin{column}{0.65\textwidth}

%% INVARIANTE
\fbox{%
  \parbox{\textwidth}{
    (B):
    \begin{itemize}
      \item Tenemos que $0 <= K < len(l)$
      \item Vale $len(vr) == K$
      \item Para j tal que $0 <= j <= K-1$, $vr[j] == l[j]$ si j es par y $vr[j] == l[j]+n$ si j es impar
    \end{itemize}
  }%
}
\\
\textcolor{blue}{Vale \I en (C)?}
\begin{itemize}
  \item Si K es par, se agrega a vr el valor de l[K]. Si es impar, se agrega a vr l[K]+n
  \item len(vr) se incrementa en 1, entonces vale $len(vr) == K + 1$
  \item i se incrementa en 1, entonces vale $i == K + 1$ (y que $i-1 == K$)
\end{itemize}

\end{column}
\end{columns}
\end{frame}

%%%%%%%%%%%%%%%%%%%%%%%%%%%%%

\begin{frame}[fragile]\frametitle{Ejercicio 1 - Correctitud}
\begin{columns}
\begin{column}{0.35\textwidth}
\lstinputlisting[language=Python]{suma-en-posiciones-impares}
\end{column}
\begin{column}{0.65\textwidth}


\textcolor{blue}{Vale \I en (C)?}
\begin{itemize}
  \item Si K es par, se agrega a vr el valor de l[K]. Si es impar, se agrega a vr l[K]+n
  \item len(vr) se incrementa en 1, entonces vale $len(vr) == K + 1$
  \item i se incrementa en 1, entonces vale $i == K + 1$ (y que $i-1 == K$)
  \item Entonces \textcolor{red}{$len(vr) = i$} \pause
  \only<2>{\item Para j tal que $\textcolor{red}{0 <= j <= K}$, $vr[j] == l[j]$ si j es par y $vr[j] == l[j]+n$ si j es impar}
  \item<3-> Para j tal que $\textcolor{red}{0 <= j <= i-1}$, $vr[j] == l[j]$ si j es par y $vr[j] == l[j]+n$ si j es impar
  \item<4-> $0 <= K < len(l)$ equivale a $0 <= K+1 <= len(l)$ y esto equivale a que $\textcolor{red}{0 <= i <= len(l)}$
\end{itemize}

\end{column}
\end{columns}
\end{frame}

%%%%%%%%%%%%%%%%%%%%%%%%%%%%%

\begin{frame}[fragile]\frametitle{Ejercicio 1 - Correctitud}
\begin{columns}
\begin{column}{0.35\textwidth}
\lstinputlisting[language=Python]{suma-en-posiciones-impares}
\end{column}
\begin{column}{0.65\textwidth}

%% INVARIANTE
\fbox{%
  \parbox{\textwidth}{
    \I:
    \begin{itemize}
      \item $0 <= i <= len(l)$
      \item $len(vr) == i$
      \item Para j tal que $0 <= j <= i-1$, $vr[j] == l[j]$ si j es par y $vr[j] == l[j]+n$ si j es impar
    \end{itemize}
  }%
}

\textcolor{green}{Vale \I en (C)!}
\begin{itemize}
  \item Entonces \textcolor{red}{$len(vr) = i$}
  \item Para j tal que $\textcolor{red}{0 <= j <= i-1}$, $vr[j] == l[j]$ si j es par y $vr[j] == l[j]+n$ si j es impar
  \item $0 <= K < len(l)$ equivale a $0 <= K+1 <= len(l)$ y esto equivale a que $\textcolor{red}{0 <= i <= len(l)}$
\end{itemize}

\end{column}
\end{columns}
\end{frame}

%%%%%%%%%%%%%%%%%%%%%%%%%%%%%

\begin{frame}[fragile]\frametitle{Ejercicio 1 - Correctitud}
\begin{columns}
\begin{column}{0.35\textwidth}
\lstinputlisting[language=Python]{suma-en-posiciones-impares}
\end{column}
\begin{column}{0.65\textwidth}

%% INVARIANTE
\fbox{%
  \parbox{\textwidth}{
    \I:
    \begin{itemize}
      \item $0 <= i <= len(l)$
      \item $len(vr) == i$
      \item Para j tal que $0 <= j <= i-1$, $vr[j] == l[j]$ si j es par y $vr[j] == l[j]+n$ si j es impar
    \end{itemize}
  }%
}
\begin{itemize}[<+->]
  \item Por \'ultimo, en (D) vale el \I y adem\'as que i == len(l)
  \item Entonces vale que para j tal que $\textcolor{red}{0 <= j <= len(l)-1}$, $vr[j] == l[j]$ si j es par y $vr[j] == l[j]+n$ si j es impar
  \item Tambi\'en que $\textcolor{red}{len(vr) == i == len(l)}$
\end{itemize}

\end{column}
\end{columns}
\pause
\textcolor{blue}{Entonces vale el devuelve: $len(vr)==len(l)$, y en toda posici\'on j entre 0 y len(l)-1:
          vr[j]==l[j] si j es par, o bien vr[j]==l[j]+n si j es impar}
\end{frame}

%%%%%%%%%%%%%%%%%%%%%%%%%%%%%

\end{document}
